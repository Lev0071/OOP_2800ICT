
\documentclass[12pt]{article}
\usepackage[utf8]{inputenc}
\usepackage{fancyvrb}
\usepackage{geometry}
\geometry{a4paper, margin=1in}
\usepackage{titlesec}
\usepackage{xcolor}
\usepackage{tcolorbox}
\titleformat{\section}[block]{\Large\bfseries}{\thesection.}{0.5em}{}
\title{Smart Pointers in C++}
\author{}
\date{}

\begin{document}
\maketitle
\tableofcontents
\newpage

\section*{Unique Pointer Demo}
\addcontentsline{toc}{section}{Unique Pointer Demo}

\subsection*{Code}
\begin{Verbatim}[fontsize=\small,frame=single]
#include <iostream>
#include <memory> // for unique_ptr and make_unique

void demoUniquePtr()
{
    int a_int = 10;

    std::unique_ptr<int> ptr = std::make_unique<int>(10);

    std::cout << "Address: " << ptr.get() << std::endl;
    std::cout << "Value: " << *ptr << std::endl;
    std::cout << "Value: " << *ptr.get() << std::endl;

    std::unique_ptr<int> ptr1 = std::make_unique<int>(42);

    // std::unique_ptr<int> ptr2 = ptr1; // ❌ Not allowed — copy constructor is deleted
    std::unique_ptr<int> ptr2 = std::move(ptr1); // ✅ Ownership transferred from ptr1 to ptr2

    if (ptr2)
        std::cout << "Transferred value: " << *ptr2 << std::endl;
}

int main()
{
    demoUniquePtr();
    return 0;
}
// g++ -std=c++11 -o unique_ptr_demo unique_ptr_demo.cpp
// g++ -std=c++14 -o unique_ptr_demo unique_ptr_demo.cpp

// ./unique_ptr_demo
\end{Verbatim}

\subsection*{Sample Output}
\begin{tcolorbox}[colback=black!5!white,colframe=black!75!white]
Sample output here...
\end{tcolorbox}
\newpage

\section*{Unique Pointer Resource Example}
\addcontentsline{toc}{section}{Unique Pointer Resource Example}

\subsection*{Code}
\begin{Verbatim}[fontsize=\small,frame=single]
#include <iostream>
#include <memory>

class Resource {
private:
    int flag = 0;

public:
    Resource(int flag = 1) : flag(flag) {
        std::cout << flag << " | Resource acquired\n";
    }

    ~Resource() {
        std::cout << flag << " | Resource destroyed\n";
    }

    friend std::ostream& operator<<(std::ostream& out, const Resource& res);
};

std::ostream& operator<<(std::ostream& out, const Resource& res) {
    out << "Resource(flag=" << res.flag << ")";
    return out;
}

void someFunction2() {
    auto ptr = std::make_unique<Resource>(2);

    std::cout << "*ptr = " << *ptr << std::endl;

    // Address of the smart pointer object itself
    std::cout << "Address of ptr (unique_ptr object): " << &ptr << std::endl;

    // Address of the actual resource managed by the smart pointer
    std::cout << "Address of resource (ptr.get()):    " << ptr.get() << std::endl;
}

int main() {
    someFunction2();
    std::cout << "End of main\n";
    return 0;
}

// g++ -std=c++14  unique_ptr_resource_demo.cpp -o unique_ptr_resource_demo
\end{Verbatim}

\subsection*{Sample Output}
\begin{tcolorbox}[colback=black!5!white,colframe=black!75!white]
Sample output here...
\end{tcolorbox}
\newpage

\section*{Unique Pointer Transfer}
\addcontentsline{toc}{section}{Unique Pointer Transfer}

\subsection*{Code}
\begin{Verbatim}[fontsize=\small,frame=single]
// unique_ptr_1.cpp
#include <iostream>
#include <memory> // for unique_ptr

void demoUniquePtr()
{
    std::unique_ptr<int> ptr = std::make_unique<int>(10);
    std::cout << "Value: " << *ptr << std::endl;

    std::unique_ptr<int> ptr1 = std::make_unique<int>(42);
    // std::unique_ptr<int> ptr2 = ptr1; // ❌ Copying not allowed
    std::unique_ptr<int> ptr2 = std::move(ptr1); // ✅ Ownership transferred

    if (ptr2)
        std::cout << "Transferred value: " << *ptr2 << std::endl;
}

int main()
{
    demoUniquePtr();
    return 0;
}
// g++ -std=c++14 -o unique_ptr_1 unique_ptr_1.cpp
// ./unique_ptr_1
\end{Verbatim}

\subsection*{Sample Output}
\begin{tcolorbox}[colback=black!5!white,colframe=black!75!white]
Sample output here...
\end{tcolorbox}
\newpage

\section*{Smart Pointer Overview}
\addcontentsline{toc}{section}{Smart Pointer Overview}

\subsection*{Code}
\begin{Verbatim}[fontsize=\small,frame=single]
#include <iostream>
#include <memory>

class Book {
public:
    Book(const std::string& name) : title(name) {
        std::cout << "📘 Book \"" << title << "\" created.\n";
    }

    ~Book() {
        std::cout << "🔥 Book \"" << title << "\" destroyed.\n";
    }

    void read() const {
        std::cout << "📖 Reading \"" << title << "\"...\n";
    }

private:
    std::string title;
};

int main() {
    std::cout << "=== Unique Pointer Section ===\n";
    {
        std::unique_ptr<Book> unique1 = std::make_unique<Book>("Unique Book 1");
        std::unique_ptr<Book> unique2 = std::make_unique<Book>("Unique Book 2");

        unique1->read();
        unique2->read();

        // Exiting this scope → both unique_ptrs are destroyed
    }

    std::cout << "\n=== Shared Pointer Section ===\n";
    {
        std::shared_ptr<Book> shared1 = std::make_shared<Book>("Shared Book");
        {
            std::shared_ptr<Book> shared2 = shared1; // shared2 also points to the same object

            shared1->read();
            shared2->read();

            std::cout << "➡️ shared2 is going out of scope...\n";
        } // shared2 is destroyed here, but shared1 still exists

        std::cout << "➡️ shared1 is going out of scope...\n";
    } // Now shared1 is destroyed → book finally deleted

    std::cout << "\n=== End of Program ===\n";
    return 0;
}
// g++ -std=c++2b smartPointers.cpp -o smartPointers
\end{Verbatim}

\subsection*{Sample Output}
\begin{tcolorbox}[colback=black!5!white,colframe=black!75!white]
Sample output here...
\end{tcolorbox}
\newpage

\section*{Shared Pointer Reference Counting}
\addcontentsline{toc}{section}{Shared Pointer Reference Counting}

\subsection*{Code}
\begin{Verbatim}[fontsize=\small,frame=single]
// shared_ptr.cpp
#include <iostream>
#include <memory>

using namespace std;

class MyClass {
public:
    MyClass() { cout << "Constructor\n"; }
    ~MyClass() { cout << "Destructor\n"; }
};

int main() {
    shared_ptr<MyClass> p1 = make_shared<MyClass>();
    cout << "p1 use count: " << p1.use_count() << endl;

    shared_ptr<MyClass> p2 = p1;
    cout << "p2 use count: " << p2.use_count() << endl;

    {
        shared_ptr<MyClass> p3 = p2;
        cout << "p3 use count: " << p3.use_count() << endl;
    }

    cout << "After p3 scope, p1 use count: " << p1.use_count() << endl;

    p2.reset();
    cout << "After p2.reset(), p1 use count: " << p1.use_count() << endl;

    p1.reset();
    cout << "After p1.reset(), p1 use count: " << p1.use_count() << endl;

    cout << "End of main\n";
    return 0;
}
// g++ -std=c++11 -o shared_ptr shared_ptr.cpp
// ./shared_ptr
\end{Verbatim}

\subsection*{Sample Output}
\begin{tcolorbox}[colback=black!5!white,colframe=black!75!white]
Sample output here...
\end{tcolorbox}
\newpage

\section*{Shared Pointer Circular Reference}
\addcontentsline{toc}{section}{Shared Pointer Circular Reference}

\subsection*{Code}
\begin{Verbatim}[fontsize=\small,frame=single]
// shared_ptr1.cpp
#include <iostream>
#include <memory>  // for shared_ptr
#include <string>

using namespace std;

class Person {
private:
    string m_name;
    shared_ptr<Person> m_partner;  // initially created empty

public:
    Person(const string& name) : m_name(name) {
        cout << m_name << " created\n";
    }

    ~Person() {
        cout << m_name << " destroyed\n";
    }

    friend bool partnerUp(shared_ptr<Person>& p1, shared_ptr<Person>& p2) {
        if (!p1 || !p2)
            return false;

        p1->m_partner = p2;
        p2->m_partner = p1;

        cout << p1->m_name << " is now partnered with " << p2->m_name << '\n';
        return true;
    }
};

void testSharedPtrCouple()
{
    auto lucy = make_shared<Person>("Lucy");
    auto ricky = make_shared<Person>("Ricky");

    partnerUp(lucy, ricky);  // Mutual ownership

    cout << "Lucy use_count: " << lucy.use_count() << endl;   // Likely 2
    cout << "Ricky use_count: " << ricky.use_count() << endl; // Likely 2
}

int main()
{
    testSharedPtrCouple();
    cout << "End of main\n";
    return 0;
}
// g++ -std=c++14  shared_ptr_circular.cpp -o shared_ptr_circular
\end{Verbatim}

\subsection*{Sample Output}
\begin{tcolorbox}[colback=black!5!white,colframe=black!75!white]
Sample output here...
\end{tcolorbox}
\newpage

\section*{Breaking Cycles with weak_ptr}
\addcontentsline{toc}{section}{Breaking Cycles with weak_ptr}

\subsection*{Code}
\begin{Verbatim}[fontsize=\small,frame=single]
#include <iostream>
#include <memory>
using namespace std;

class Person {
private:
    string m_name;
    weak_ptr<Person> m_partner;  // initially created empty

public:
    Person(const string& name) : m_name(name) {
        cout << m_name << " created\n";
    }

    ~Person() {
        cout << m_name << " destroyed\n";
    }

    friend bool partnerUp(shared_ptr<Person>& p1, shared_ptr<Person>& p2) {
        if (!p1 || !p2)
            return false;

        p1->m_partner = p2;
        p2->m_partner = p1;

        cout << p1->m_name << " is now partnered with " << p2->m_name << '\n';
        return true;
    }
};
void testSharedPtrCouple()
{
    auto lucy = make_shared<Person>("Lucy");
    auto ricky = make_shared<Person>("Ricky");

    partnerUp(lucy, ricky);  // Mutual ownership

    cout << "Lucy use_count: " << lucy.use_count() << endl;   // Likely 2
    cout << "Ricky use_count: " << ricky.use_count() << endl; // Likely 2
}

int main()
{
    testSharedPtrCouple();
    cout << "End of main\n";
    return 0;
}
// g++ -std=c++14  shared_ptr_weak_circular.cpp -o shared_ptr_weak_circular

// ✅ If two objects mutually reference each other with a **weak_ptr**, here’s what happens:
// 📌 Key properties of std::weak_ptr:

//     It does not increase the reference count.

//     It’s a non-owning observer.

//     It becomes useful only if there's at least one std::shared_ptr keeping the object alive.


// ✅ Why this works:

//     The only owners are a and b — both are shared_ptrs in main.

//     a->bptr and b->aptr are weak_ptrs — they don’t increase use_count().

//     So when a and b go out of scope, both objects are properly destroyed — no circular dependency.

// ==============================================
// #include <iostream>
// #include <memory>
// using namespace std;

// class B; // Forward declaration

// class A {
// public:
//     weak_ptr<B> bptr;  // weak reference to B
//     ~A() { cout << "A destroyed\n"; }
// };

// class B {
// public:
//     weak_ptr<A> aptr;  // weak reference to A
//     ~B() { cout << "B destroyed\n"; }
// };

// int main() {
//     shared_ptr<A> a = make_shared<A>();
//     shared_ptr<B> b = make_shared<B>();

//     a->bptr = b;  // no ownership added
//     b->aptr = a;  // no ownership added

//     cout << "a use_count: " << a.use_count() << endl;  // 1
//     cout << "b use_count: " << b.use_count() << endl;  // 1
// }
\end{Verbatim}

\subsection*{Sample Output}
\begin{tcolorbox}[colback=black!5!white,colframe=black!75!white]
Sample output here...
\end{tcolorbox}
\newpage

\section*{Accessing Partner via weak_ptr.lock()}
\addcontentsline{toc}{section}{Accessing Partner via weak_ptr.lock()}

\subsection*{Code}
\begin{Verbatim}[fontsize=\small,frame=single]
// weak_to_share.cpp
#include <iostream>
#include <memory>     // for shared_ptr and weak_ptr
#include <string>
using namespace std;

class Person {
private:
    string m_name;
    weak_ptr<Person> m_partner;  // note: This is now a weak_ptr

public:
    Person(const string &name) : m_name(name) {
        cout << m_name << " created\n";
    }

    ~Person() {
        cout << m_name << " destroyed\n";
    }

    friend bool partnerUp(shared_ptr<Person> &p1, shared_ptr<Person> &p2) {
        if (!p1 || !p2)
            return false;

        p1->m_partner = p2;
        p2->m_partner = p1;

        cout << p1->m_name << " is now partnered with " << p2->m_name << '\n';
        return true;
    }

    // use lock() to convert weak_ptr to shared_ptr
    const shared_ptr<Person> getPartner() const {
        return m_partner.lock();
    }

    const string &getName() const {
        return m_name;
    }
};

int main() {
    auto lucy{make_shared<Person>("Lucy")};
    auto ricky{make_shared<Person>("Ricky")};

    partnerUp(lucy, ricky);

    auto partner = ricky->getPartner();  // get shared_ptr to Ricky's partner
    cout << ricky->getName() << "'s partner is: " << partner->getName() << '\n';

    return 0;
}
// g++ -std=c++14  weak_to_share.cpp -o weak_to_share
\end{Verbatim}

\subsection*{Sample Output}
\begin{tcolorbox}[colback=black!5!white,colframe=black!75!white]
Sample output here...
\end{tcolorbox}
\newpage
\end{document}